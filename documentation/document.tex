\documentclass{article}
\usepackage[left=3cm, right=3cm]{geometry}
\usepackage[italian]{babel}
%for clickable table of content
\usepackage{hyperref}
\hypersetup{
	colorlinks,
	citecolor=black,
	filecolor=black,
	linkcolor=black,
	urlcolor=black
}
%opening
\title{Progetto finale di Reti Logiche}
\author{Roland Reylander [10539438]}
\date{1 Aprile 2019}
\begin{document}

\maketitle
\tableofcontents
\pagebreak

\section{Specifiche di progetto}
Lo scopo del progetto \`e la realizzazione di un componente hardware usando il linguaggio VHDL che risolve il seguente problema: in uno spazio quadrato di 256x256 vengono posizionati 8 centroidi e si vuole trovare il centroide o i centroidi più vicini ad un punto dato.
\newline
Il componente dovr\`{a} interfacciarsi con una memoria RAM dalla quale legge i dati di input e scrive il risultato della computazione. La memoria RAM di input contiene la maschera dei centroidi da considerare, le coordinate dei centroidi, le coordinate del punto dal quale calcolare la distanza e un inidirizzo dedicato al risultato, cioè la maschera di uscita.
\newline
\newline

\renewcommand{\arraystretch}{1.5}
\begin{tabular}{ |c|c|c| }
	\hline
	INDIRIZZO RAM & CONTENUTO \\ 
	\hline
	0 & maschera dei centroidi \\
	\hline
	1 & coordinata X del centroide \\
	\hline
	2 & coordinata Y del centroide \\
	\hline
	\vdots & \vdots \\
	\hline
	17 & coordinata X del punto da considerare \\
	\hline
	18 & coordinata Y del punto da considerare \\
	\hline
	19 & maschera di output del risultato \\
	\hline
	\vdots & indirizzi non utili \\
	\hline
\end{tabular}
\newline



\section{Scelte progettuali}
Per affrontare la risoluzione di questo problema \`{e} stata pensata una FSM con i seguenti stati: 
\begin{itemize}
	\item \textbf{reset}: tutti i signal vengono inizializzati ad un valore di inizio;
	\item \textbf{changeAddress}: in base al valore che assume \texttt{cnt} (cio\`{e} \texttt{readMask, readXPoint, readYPoint, readXCoord, readYCoord}) viene aumentato l'indirizzo per la prossima lettura della RAM;
	\item \textbf{waitClock}: la lettura dalla memoria richiede 2 ns e quindi non \`{e} immediata. In questo stato si aspetta un ciclo di clock affinch\'{e} il valore \texttt{i\_data} assuma il valore desiderato che verr\`{a} poi usato nello stato successivo;
	\item \textbf{readData}: in base al valore di \texttt{cnt} in questo stato vengono salvati su dei signal dedicati i valori della maschera e delle coordinate del punto. Successivamente verranno usati i signal \texttt{xAddress} e \texttt{yAddress} per memorizzare le coordinate dei centroidi;
	\item \textbf{calcDistance}: viene calcolata la distanza e memorizzata in \texttt{tempDistance};
	\item \textbf{compareDistance}: la distanza calcolata al punto precedente viene confrontata con \texttt{bestDistance} in modo da tenere solo i centroidi pi\`{u} vicini;
	\item \textbf{sendMask}: \texttt{t\_data} assume il valore del risultato finale e l'indirizzo da mandare alla RAM è il 19° e \texttt{o\_we} viene settato a $1$ per la scrittura del risultato;
	\item \textbf{load}: segnale di \texttt{o\_done} viene messo a $1$ per indicare la fine dell'elaborazione e l'avvenuta scrittura del risultato;
	\item \textbf{last}: \texttt{o\_done} torna a $0$ e il componente torna allo stato di \texttt{reset} pronto per ricevere il prossimo segnale di start.
	
\end{itemize}
Dopo la lettura della maschera dalla RAM, nello stato \texttt{changeAddress} viene scandita la maschera dei centroidi da considerare tramite il signal \texttt{maskPos} in modo da passare alla lettura delle coordinate dei centroidi in caso il bit sia 1 altrimenti verr\`{a} letto il prossimo bit della maschera.
\newline
I signal \texttt{tempDistance} e \texttt{bestDistance} sono entrambi \texttt{std\_logic\_vector(8 DOWNTO 0)} dato che la distanza massima delle coordinate X e Y dei centroidi \`{e} $255+255=510$ e necessita quindi di 8 bit. Inoltre \texttt{bestDistance} viene inizializzata nello stato di \textit{reset} con \texttt{OTHERS => '1'} cio\`{e} $511$ che \`{e} maggiore della massima distanza possibile dei centroidi e quindi permette il corretto funzionamento dell'algoritmo.


\section{Risultati dei test}
\section{Risultati della sintesi}




\end{document}